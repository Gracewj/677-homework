\documentclass[]{article}
\usepackage{lmodern}
\usepackage{amssymb,amsmath}
\usepackage{ifxetex,ifluatex}
\usepackage{fixltx2e} % provides \textsubscript
\ifnum 0\ifxetex 1\fi\ifluatex 1\fi=0 % if pdftex
  \usepackage[T1]{fontenc}
  \usepackage[utf8]{inputenc}
\else % if luatex or xelatex
  \ifxetex
    \usepackage{mathspec}
  \else
    \usepackage{fontspec}
  \fi
  \defaultfontfeatures{Ligatures=TeX,Scale=MatchLowercase}
\fi
% use upquote if available, for straight quotes in verbatim environments
\IfFileExists{upquote.sty}{\usepackage{upquote}}{}
% use microtype if available
\IfFileExists{microtype.sty}{%
\usepackage{microtype}
\UseMicrotypeSet[protrusion]{basicmath} % disable protrusion for tt fonts
}{}
\usepackage[margin=1in]{geometry}
\usepackage{hyperref}
\hypersetup{unicode=true,
            pdftitle={Homework 4},
            pdfauthor={Wenjia Xie},
            pdfborder={0 0 0},
            breaklinks=true}
\urlstyle{same}  % don't use monospace font for urls
\usepackage{color}
\usepackage{fancyvrb}
\newcommand{\VerbBar}{|}
\newcommand{\VERB}{\Verb[commandchars=\\\{\}]}
\DefineVerbatimEnvironment{Highlighting}{Verbatim}{commandchars=\\\{\}}
% Add ',fontsize=\small' for more characters per line
\usepackage{framed}
\definecolor{shadecolor}{RGB}{248,248,248}
\newenvironment{Shaded}{\begin{snugshade}}{\end{snugshade}}
\newcommand{\KeywordTok}[1]{\textcolor[rgb]{0.13,0.29,0.53}{\textbf{#1}}}
\newcommand{\DataTypeTok}[1]{\textcolor[rgb]{0.13,0.29,0.53}{#1}}
\newcommand{\DecValTok}[1]{\textcolor[rgb]{0.00,0.00,0.81}{#1}}
\newcommand{\BaseNTok}[1]{\textcolor[rgb]{0.00,0.00,0.81}{#1}}
\newcommand{\FloatTok}[1]{\textcolor[rgb]{0.00,0.00,0.81}{#1}}
\newcommand{\ConstantTok}[1]{\textcolor[rgb]{0.00,0.00,0.00}{#1}}
\newcommand{\CharTok}[1]{\textcolor[rgb]{0.31,0.60,0.02}{#1}}
\newcommand{\SpecialCharTok}[1]{\textcolor[rgb]{0.00,0.00,0.00}{#1}}
\newcommand{\StringTok}[1]{\textcolor[rgb]{0.31,0.60,0.02}{#1}}
\newcommand{\VerbatimStringTok}[1]{\textcolor[rgb]{0.31,0.60,0.02}{#1}}
\newcommand{\SpecialStringTok}[1]{\textcolor[rgb]{0.31,0.60,0.02}{#1}}
\newcommand{\ImportTok}[1]{#1}
\newcommand{\CommentTok}[1]{\textcolor[rgb]{0.56,0.35,0.01}{\textit{#1}}}
\newcommand{\DocumentationTok}[1]{\textcolor[rgb]{0.56,0.35,0.01}{\textbf{\textit{#1}}}}
\newcommand{\AnnotationTok}[1]{\textcolor[rgb]{0.56,0.35,0.01}{\textbf{\textit{#1}}}}
\newcommand{\CommentVarTok}[1]{\textcolor[rgb]{0.56,0.35,0.01}{\textbf{\textit{#1}}}}
\newcommand{\OtherTok}[1]{\textcolor[rgb]{0.56,0.35,0.01}{#1}}
\newcommand{\FunctionTok}[1]{\textcolor[rgb]{0.00,0.00,0.00}{#1}}
\newcommand{\VariableTok}[1]{\textcolor[rgb]{0.00,0.00,0.00}{#1}}
\newcommand{\ControlFlowTok}[1]{\textcolor[rgb]{0.13,0.29,0.53}{\textbf{#1}}}
\newcommand{\OperatorTok}[1]{\textcolor[rgb]{0.81,0.36,0.00}{\textbf{#1}}}
\newcommand{\BuiltInTok}[1]{#1}
\newcommand{\ExtensionTok}[1]{#1}
\newcommand{\PreprocessorTok}[1]{\textcolor[rgb]{0.56,0.35,0.01}{\textit{#1}}}
\newcommand{\AttributeTok}[1]{\textcolor[rgb]{0.77,0.63,0.00}{#1}}
\newcommand{\RegionMarkerTok}[1]{#1}
\newcommand{\InformationTok}[1]{\textcolor[rgb]{0.56,0.35,0.01}{\textbf{\textit{#1}}}}
\newcommand{\WarningTok}[1]{\textcolor[rgb]{0.56,0.35,0.01}{\textbf{\textit{#1}}}}
\newcommand{\AlertTok}[1]{\textcolor[rgb]{0.94,0.16,0.16}{#1}}
\newcommand{\ErrorTok}[1]{\textcolor[rgb]{0.64,0.00,0.00}{\textbf{#1}}}
\newcommand{\NormalTok}[1]{#1}
\usepackage{graphicx,grffile}
\makeatletter
\def\maxwidth{\ifdim\Gin@nat@width>\linewidth\linewidth\else\Gin@nat@width\fi}
\def\maxheight{\ifdim\Gin@nat@height>\textheight\textheight\else\Gin@nat@height\fi}
\makeatother
% Scale images if necessary, so that they will not overflow the page
% margins by default, and it is still possible to overwrite the defaults
% using explicit options in \includegraphics[width, height, ...]{}
\setkeys{Gin}{width=\maxwidth,height=\maxheight,keepaspectratio}
\IfFileExists{parskip.sty}{%
\usepackage{parskip}
}{% else
\setlength{\parindent}{0pt}
\setlength{\parskip}{6pt plus 2pt minus 1pt}
}
\setlength{\emergencystretch}{3em}  % prevent overfull lines
\providecommand{\tightlist}{%
  \setlength{\itemsep}{0pt}\setlength{\parskip}{0pt}}
\setcounter{secnumdepth}{0}
% Redefines (sub)paragraphs to behave more like sections
\ifx\paragraph\undefined\else
\let\oldparagraph\paragraph
\renewcommand{\paragraph}[1]{\oldparagraph{#1}\mbox{}}
\fi
\ifx\subparagraph\undefined\else
\let\oldsubparagraph\subparagraph
\renewcommand{\subparagraph}[1]{\oldsubparagraph{#1}\mbox{}}
\fi

%%% Use protect on footnotes to avoid problems with footnotes in titles
\let\rmarkdownfootnote\footnote%
\def\footnote{\protect\rmarkdownfootnote}

%%% Change title format to be more compact
\usepackage{titling}

% Create subtitle command for use in maketitle
\newcommand{\subtitle}[1]{
  \posttitle{
    \begin{center}\large#1\end{center}
    }
}

\setlength{\droptitle}{-2em}

  \title{Homework 4}
    \pretitle{\vspace{\droptitle}\centering\huge}
  \posttitle{\par}
    \author{Wenjia Xie}
    \preauthor{\centering\large\emph}
  \postauthor{\par}
      \predate{\centering\large\emph}
  \postdate{\par}
    \date{March 8, 2019}


\begin{document}
\maketitle

\section{Problem One}\label{problem-one}

\subsection{(1)}\label{section}

\begin{Shaded}
\begin{Highlighting}[]
\NormalTok{uni <-}\StringTok{ }\KeywordTok{c}\NormalTok{(}\FloatTok{0.42}\NormalTok{,}\FloatTok{0.38}\NormalTok{,}\FloatTok{0.48}\NormalTok{,}\FloatTok{0.11}\NormalTok{,}\FloatTok{0.30}\NormalTok{,}\FloatTok{0.06}\NormalTok{,}\FloatTok{0.78}\NormalTok{,}\FloatTok{0.35}\NormalTok{,}\FloatTok{0.29}\NormalTok{,}\FloatTok{0.23}\NormalTok{,}\FloatTok{0.88}\NormalTok{,}\FloatTok{0.71}\NormalTok{,}\FloatTok{0.16}\NormalTok{, }\FloatTok{0.79}\NormalTok{,}\FloatTok{0.01}\NormalTok{,}\FloatTok{0.40}\NormalTok{,}\FloatTok{0.57}\NormalTok{,}\FloatTok{0.22}\NormalTok{,}\FloatTok{0.75}\NormalTok{,}\FloatTok{0.41}\NormalTok{,}\FloatTok{0.90}\NormalTok{,}\FloatTok{0.66}\NormalTok{,}\FloatTok{0.08}\NormalTok{,}\FloatTok{0.82}\NormalTok{,}\FloatTok{0.09}\NormalTok{)}
\KeywordTok{ks.test}\NormalTok{(uni,}\StringTok{"punif"}\NormalTok{)}
\end{Highlighting}
\end{Shaded}

\begin{verbatim}
## 
##  One-sample Kolmogorov-Smirnov test
## 
## data:  uni
## D = 0.18, p-value = 0.3501
## alternative hypothesis: two-sided
\end{verbatim}

\begin{verbatim}
                                                                        Since the p-value here is 0.3501, we can not reject the null hypothesis. The data is distributed as a uniform distribution.                 
\end{verbatim}

\subsection{(2)}\label{section-1}

\begin{Shaded}
\begin{Highlighting}[]
\NormalTok{x <-}\StringTok{ }\KeywordTok{seq}\NormalTok{(}\OperatorTok{-}\DecValTok{1}\NormalTok{,}\DecValTok{1}\NormalTok{,}\FloatTok{0.01}\NormalTok{)}
\NormalTok{fx <-}\StringTok{ }\KeywordTok{ifelse}\NormalTok{(x }\OperatorTok{>}\StringTok{ }\DecValTok{0} \OperatorTok{&}\StringTok{ }\NormalTok{x }\OperatorTok{<=}\FloatTok{0.5}\NormalTok{, }\DecValTok{2}\OperatorTok{/}\DecValTok{3}\NormalTok{,}
   \KeywordTok{ifelse}\NormalTok{(x }\OperatorTok{>}\StringTok{ }\FloatTok{0.5} \OperatorTok{&}\StringTok{ }\NormalTok{x }\OperatorTok{<}\StringTok{ }\DecValTok{1}\NormalTok{,  }\FloatTok{0.5}\NormalTok{, }\DecValTok{0}\NormalTok{))}
\KeywordTok{ks.test}\NormalTok{(uni,fx)}
\end{Highlighting}
\end{Shaded}

\begin{verbatim}
## Warning in ks.test(uni, fx): cannot compute exact p-value with ties
\end{verbatim}

\begin{verbatim}
## 
##  Two-sample Kolmogorov-Smirnov test
## 
## data:  uni and fx
## D = 0.50746, p-value = 2.127e-05
## alternative hypothesis: two-sided
\end{verbatim}

The p-value is so small that we reject the null hypothesis.Thus,the data
is not distributed as fx.

\subsection{(3)}\label{section-2}

\section{Problem Two}\label{problem-two}

\begin{Shaded}
\begin{Highlighting}[]
\NormalTok{normal <-}\StringTok{ }\KeywordTok{c}\NormalTok{(}\FloatTok{25.088}\NormalTok{,}\FloatTok{26.615}\NormalTok{,}\FloatTok{25.468}\NormalTok{,}\FloatTok{27.453}\NormalTok{,}\FloatTok{23.845}\NormalTok{,}
\FloatTok{25.996}\NormalTok{,}\FloatTok{26.516}\NormalTok{,}\FloatTok{28.240}\NormalTok{,}\FloatTok{25.980}\NormalTok{,}\FloatTok{30.432}\NormalTok{,}
\FloatTok{26.560}\NormalTok{,}\FloatTok{25.844}\NormalTok{,}\FloatTok{26.964}\NormalTok{,}\FloatTok{23.382}\NormalTok{,}\FloatTok{25.282}\NormalTok{,}
\FloatTok{24.432}\NormalTok{,}\FloatTok{23.593}\NormalTok{,}\FloatTok{24.644}\NormalTok{,}\FloatTok{26.849}\NormalTok{,}\FloatTok{26.801}\NormalTok{,}
\FloatTok{26.303}\NormalTok{,}\FloatTok{23.016}\NormalTok{,}\FloatTok{27.378}\NormalTok{,}\FloatTok{25.351}\NormalTok{,}\FloatTok{23.601}\NormalTok{,}
\FloatTok{24.317}\NormalTok{,}\FloatTok{29.778}\NormalTok{,}\FloatTok{29.585}\NormalTok{,}\FloatTok{22.147}\NormalTok{,}\FloatTok{28.352}\NormalTok{,}
\FloatTok{29.263}\NormalTok{,}\FloatTok{27.924}\NormalTok{,}\FloatTok{21.579}\NormalTok{,}\FloatTok{25.320}\NormalTok{,}\FloatTok{28.129}\NormalTok{,}
\FloatTok{28.478}\NormalTok{,}\FloatTok{23.896}\NormalTok{,}\FloatTok{26.020}\NormalTok{,}\FloatTok{23.750}\NormalTok{,}\FloatTok{24.904}\NormalTok{,}
\FloatTok{24.078}\NormalTok{,}\FloatTok{27.228}\NormalTok{,}\FloatTok{27.433}\NormalTok{,}\FloatTok{23.341}\NormalTok{,}\FloatTok{28.923}\NormalTok{,}
\FloatTok{24.466}\NormalTok{,}\FloatTok{25.153}\NormalTok{,}\FloatTok{25.893}\NormalTok{,}\FloatTok{26.796}\NormalTok{,}\FloatTok{24.743}\NormalTok{)}
\KeywordTok{ks.test}\NormalTok{(normal,}\StringTok{"pnorm"}\NormalTok{,}\DecValTok{26}\NormalTok{,}\DecValTok{2}\NormalTok{)}
\end{Highlighting}
\end{Shaded}

\begin{verbatim}
## 
##  One-sample Kolmogorov-Smirnov test
## 
## data:  normal
## D = 0.06722, p-value = 0.9663
## alternative hypothesis: two-sided
\end{verbatim}

\begin{Shaded}
\begin{Highlighting}[]
\KeywordTok{qqnorm}\NormalTok{(normal)}
\KeywordTok{qqline}\NormalTok{(normal, }\DataTypeTok{col =} \StringTok{"steelblue"}\NormalTok{, }\DataTypeTok{lwd =} \DecValTok{2}\NormalTok{)}
\end{Highlighting}
\end{Shaded}

\includegraphics{homework_files/figure-latex/unnamed-chunk-3-1.pdf}

\begin{Shaded}
\begin{Highlighting}[]
\KeywordTok{hist}\NormalTok{(normal)}
\end{Highlighting}
\end{Shaded}

\includegraphics{homework_files/figure-latex/unnamed-chunk-3-2.pdf}

\begin{Shaded}
\begin{Highlighting}[]
\KeywordTok{plot}\NormalTok{(}\KeywordTok{density}\NormalTok{(normal))}
\end{Highlighting}
\end{Shaded}

\includegraphics{homework_files/figure-latex/unnamed-chunk-3-3.pdf}

The p-value is large enough and we do not have sufficient evidence to
rejuct the null hypothesis.From the qqplot,histgram and density plot, we
can also draw the same conclusion that these data come from normal
distribution.

\section{Problem Three}\label{problem-three}

\begin{Shaded}
\begin{Highlighting}[]
\NormalTok{X <-}\StringTok{ }\KeywordTok{c}\NormalTok{(}\FloatTok{0.61}\NormalTok{,}\FloatTok{0.29}\NormalTok{,}\FloatTok{0.06}\NormalTok{,}\FloatTok{0.59}\NormalTok{,−}\FloatTok{1.73}\NormalTok{,}
\NormalTok{−}\FloatTok{0.74}\NormalTok{,}\FloatTok{0.51}\NormalTok{,−}\FloatTok{0.56}\NormalTok{,−}\FloatTok{0.39}\NormalTok{,}\FloatTok{1.64}\NormalTok{,}
\FloatTok{0.05}\NormalTok{,−}\FloatTok{0.06}\NormalTok{,}\FloatTok{0.64}\NormalTok{,−}\FloatTok{0.82}\NormalTok{,}\FloatTok{0.31}\NormalTok{,}
\FloatTok{1.77}\NormalTok{,}\FloatTok{1.09}\NormalTok{,−}\FloatTok{1.28}\NormalTok{,}\FloatTok{2.36}\NormalTok{,}\FloatTok{1.31}\NormalTok{,}
\FloatTok{1.05}\NormalTok{,−}\FloatTok{0.32}\NormalTok{,−}\FloatTok{0.40}\NormalTok{,}\FloatTok{1.06}\NormalTok{,−}\FloatTok{2.47}\NormalTok{)}

\NormalTok{Y <-}\StringTok{ }\KeywordTok{c}\NormalTok{(}\FloatTok{2.20}\NormalTok{,}\FloatTok{1.66}\NormalTok{,}\FloatTok{1.38}\NormalTok{,}\FloatTok{0.20}\NormalTok{,}
\FloatTok{0.36}\NormalTok{,}\FloatTok{0.00}\NormalTok{,}\FloatTok{0.96}\NormalTok{,}\FloatTok{1.56}\NormalTok{,}
\FloatTok{0.44}\NormalTok{,}\FloatTok{1.50}\NormalTok{,−}\FloatTok{0.30}\NormalTok{,}\FloatTok{0.66}\NormalTok{,}
\FloatTok{2.31}\NormalTok{,}\FloatTok{3.29}\NormalTok{,−}\FloatTok{0.27}\NormalTok{,−}\FloatTok{0.37}\NormalTok{,}
\FloatTok{0.38}\NormalTok{,}\FloatTok{0.70}\NormalTok{,}\FloatTok{0.52}\NormalTok{,−}\FloatTok{0.71}\NormalTok{)}

\KeywordTok{ks.test}\NormalTok{(X,Y)}
\end{Highlighting}
\end{Shaded}

\begin{verbatim}
## 
##  Two-sample Kolmogorov-Smirnov test
## 
## data:  X and Y
## D = 0.27, p-value = 0.3357
## alternative hypothesis: two-sided
\end{verbatim}

\begin{Shaded}
\begin{Highlighting}[]
\KeywordTok{ks.test}\NormalTok{(X}\OperatorTok{+}\DecValTok{2}\NormalTok{,Y)}
\end{Highlighting}
\end{Shaded}

\begin{verbatim}
## Warning in ks.test(X + 2, Y): cannot compute exact p-value with ties
\end{verbatim}

\begin{verbatim}
## 
##  Two-sample Kolmogorov-Smirnov test
## 
## data:  X + 2 and Y
## D = 0.56, p-value = 0.001881
## alternative hypothesis: two-sided
\end{verbatim}

Using ks.test, we can find that X and Y are from the same distribution.
However, X+2 and Y are not from the same distribution.

\section{Problem Four}\label{problem-four}

\begin{Shaded}
\begin{Highlighting}[]
\NormalTok{data4 <-}\StringTok{ }\KeywordTok{readRDS}\NormalTok{(}\StringTok{"norm_sample.Rdata"}\NormalTok{)}
\KeywordTok{ks.test}\NormalTok{(data4,}\StringTok{"pnorm"}\NormalTok{)}
\end{Highlighting}
\end{Shaded}

\begin{verbatim}
## 
##  One-sample Kolmogorov-Smirnov test
## 
## data:  data4
## D = 0.17724, p-value = 0.3683
## alternative hypothesis: two-sided
\end{verbatim}

\begin{Shaded}
\begin{Highlighting}[]
\KeywordTok{set.seed}\NormalTok{(}\DecValTok{1}\NormalTok{)}
\KeywordTok{ecdf}\NormalTok{(data4)}
\end{Highlighting}
\end{Shaded}

\begin{verbatim}
## Empirical CDF 
## Call: ecdf(data4)
##  x[1:25] =  -2.46,  -2.11,  -1.23,  ...,   1.64,   1.76
\end{verbatim}

\begin{Shaded}
\begin{Highlighting}[]
\NormalTok{standnorm <-}\KeywordTok{rnorm}\NormalTok{(}\DataTypeTok{n =} \KeywordTok{length}\NormalTok{(data4),}\DataTypeTok{mean =} \DecValTok{0}\NormalTok{,}\DataTypeTok{sd =} \DecValTok{1}\NormalTok{)}
\NormalTok{diff <-}\StringTok{ }\KeywordTok{sort}\NormalTok{(data4)}\OperatorTok{-}\KeywordTok{sort}\NormalTok{(standnorm)}
\NormalTok{D <-}\StringTok{ }\KeywordTok{max}\NormalTok{(}\KeywordTok{abs}\NormalTok{(diff))}
\end{Highlighting}
\end{Shaded}

\section{problem Five}\label{problem-five}

\begin{Shaded}
\begin{Highlighting}[]
\NormalTok{fiji<-}\KeywordTok{read.table}\NormalTok{(}\StringTok{"fijiquakes.dat"}\NormalTok{,}\DataTypeTok{header =}\NormalTok{ T)}
\NormalTok{mag<-fiji}\OperatorTok{$}\NormalTok{mag}
\NormalTok{Fn <-}\StringTok{ }\KeywordTok{ecdf}\NormalTok{(mag)}
\KeywordTok{library}\NormalTok{(Hmisc)}
\end{Highlighting}
\end{Shaded}

\begin{verbatim}
## Loading required package: lattice
\end{verbatim}

\begin{verbatim}
## Loading required package: survival
\end{verbatim}

\begin{verbatim}
## Loading required package: Formula
\end{verbatim}

\begin{verbatim}
## Loading required package: ggplot2
\end{verbatim}

\begin{verbatim}
## 
## Attaching package: 'Hmisc'
\end{verbatim}

\begin{verbatim}
## The following objects are masked from 'package:base':
## 
##     format.pval, units
\end{verbatim}

\begin{Shaded}
\begin{Highlighting}[]
\NormalTok{total<-}\KeywordTok{sum}\NormalTok{( (mag}\OperatorTok{<=}\FloatTok{4.9}\NormalTok{) }\OperatorTok{&}\StringTok{ }\NormalTok{(mag}\OperatorTok{>}\FloatTok{4.3}\NormalTok{))}
\KeywordTok{binconf}\NormalTok{(total,}\KeywordTok{length}\NormalTok{(mag),}\DataTypeTok{method=}\StringTok{"wilson"}\NormalTok{,}\FloatTok{0.05}\NormalTok{)}
\end{Highlighting}
\end{Shaded}

\begin{verbatim}
##  PointEst     Lower     Upper
##     0.526 0.4950118 0.5567892
\end{verbatim}

The 95\% for F(4.9) − F(4.3) is {[}0.50,0.56{]}.

\begin{Shaded}
\begin{Highlighting}[]
\NormalTok{faith<-}\KeywordTok{read.table}\NormalTok{(}\StringTok{"faithful.dat"}\NormalTok{,}\DataTypeTok{skip =} \DecValTok{25}\NormalTok{)}
\NormalTok{waiting<-faith}\OperatorTok{$}\NormalTok{waiting}
\NormalTok{avg<-}\KeywordTok{mean}\NormalTok{(waiting)}
\NormalTok{var<-}\KeywordTok{var}\NormalTok{(waiting)}
\NormalTok{n<-}\KeywordTok{length}\NormalTok{(waiting)}
\NormalTok{L<-}\KeywordTok{round}\NormalTok{(avg}\OperatorTok{-}\KeywordTok{qnorm}\NormalTok{(}\FloatTok{0.95}\NormalTok{)}\OperatorTok{*}\KeywordTok{sqrt}\NormalTok{(var}\OperatorTok{/}\NormalTok{n),}\DecValTok{2}\NormalTok{)}
\NormalTok{U<-}\KeywordTok{round}\NormalTok{(avg}\OperatorTok{+}\KeywordTok{qnorm}\NormalTok{(}\FloatTok{0.95}\NormalTok{)}\OperatorTok{*}\KeywordTok{sqrt}\NormalTok{(var}\OperatorTok{/}\NormalTok{n),}\DecValTok{2}\NormalTok{)}

\KeywordTok{print}\NormalTok{(}\KeywordTok{paste}\NormalTok{(}\StringTok{"the 90% CI for mean waiting time is:["}\NormalTok{,L,}\StringTok{","}\NormalTok{,U,}\StringTok{"]"}\NormalTok{))}
\end{Highlighting}
\end{Shaded}

\begin{verbatim}
## [1] "the 90% CI for mean waiting time is:[ 69.54 , 72.25 ]"
\end{verbatim}

\begin{Shaded}
\begin{Highlighting}[]
\KeywordTok{median}\NormalTok{(waiting)}
\end{Highlighting}
\end{Shaded}

\begin{verbatim}
## [1] 76
\end{verbatim}

For the faithful data, the 90 percent confidence interval for the mean
waiting time is {[}69.54,72.25{]}. The median of the waiting time is 76.


\end{document}
